\documentclass[a4paper,12pt,twoside]{article}

\usepackage[ngerman,english]{babel}

%%% FOR DISPLAYING CODE
\usepackage[procnames]{listings}
\usepackage{xcolor}

%%% MARGINS
\usepackage{geometry}
% INFO: https://www.overleaf.com/learn/latex/Page_size_and_margins
\geometry{
	a4paper,
%	total={170mm,257mm},
	left=25mm,
	top=25mm,
	textwidth=16cm,
	textheight=23cm,
%	right=25mm,
%	bottom=40mm,$
}

%%% SPACING
%\usepackage{setspace} 
%\onehalfspacing

\usepackage{siunitx} % for writing numbers with units % e.g. \SI{6.3e8}{km}, \ang{62.59}, \num{100}'s \si{km}, \SI{53}{\degree}
\usepackage{amsmath,amssymb}
\usepackage{graphicx}
\usepackage{datetime} 
\usepackage{color}
\usepackage{float} %without it, tables position more or less random, with this an writing \begin{table}[H] the table appears right where it is defined!
\usepackage{mathptmx}
\usepackage{enumitem}

\usepackage[T1]{fontenc}
\usepackage[utf8]{inputenc}

\usepackage{pgfplots} % for plotting functions
\pgfplotsset{compat=1.16}

%%% HEADER and FOOTER
%\usepackage[automark,headsepline]{scrpage2}
%\ihead[]{\headmark}
%\chead[]{}
%\ohead[]{\pagemark}
%\ifoot[]{}
%\cfoot[\pagemark]{}
%\ofoot[]{}
%\pagestyle{scrheadings}

\usepackage{lipsum}

%% LINE SPACING
\usepackage{setspace}
\setstretch{1.5}

%%% HYPERLINKS
\usepackage{hyperref} % muss sich vor cite stuff befinden

%% SET IMAGE PATH (where images are stored)
\graphicspath{{figures/}}

%%% CITE
\usepackage{apacite}
\bibliographystyle{apacite}

%%%%%%%%%%%%%%%%%%%%%%%%%%%%%%%%%%
% DEFINE COLORS
\definecolor{red}{rgb}{0.6,0,0} 
\definecolor{blue}{rgb}{0,0,0.6}
\definecolor{green}{rgb}{0,0.8,0}
\definecolor{cyan}{rgb}{0.0,0.6,0.6}

\definecolor{mypink}{rgb}{0.753,0.000,0.890}
\definecolor{myblue}{rgb}{0.078,0.000,1.000}
\definecolor{mybluedark}{rgb}{0.004,0.024,0.525} \definecolor{mygreen}{rgb}{0.000,0.514,0.000}
\definecolor{myreddark}{rgb}{0.698,0.000,0.008}
\definecolor{mycyan}{rgb}{0.000,0.506,0.612}
\definecolor{mybrown}{rgb}{0.494,0.365,0.090}

%%% DISPLAY CODE
\usepackage{listings}
\newcommand\pythonstyle{\lstset{
    language=Python,
	tabsize=4,
	basicstyle=\normalsize\sffamily,
	numberstyle=\color{gray},
	stringstyle=\color{myreddark},
    commentstyle=\color{mygreen},
    % KEYWORDS
    % main keywords
	keywordstyle=\normalsize\color{myblue},%\bfseries,
    % add keywords (main blue)
    emph={False,None,True,self,TODO},
    emphstyle={\color{myblue}},
    % pink emph
    emph={[2]assert,break,continue,del,elif ,else,except,finally,for,from,global,if,import,in,pass,raise,return,try,while,with,yield},
    emphstyle={[2]\color{mypink}},%\bfseries,
    %dark blue emph
    emph={[3]execfile,reduce,xrange},
    emphstyle={[3]\color{mybluedark}},
    % brown emph
    emph={[4]exec,print,isinstance,zip,enumerate,reversed,len,repr},
    emphstyle={[4]\color{mybrown}},
    % cyan emph
    emph={[5]object,type,list,set,dict,tuple,str,super},
    emphstyle={[5]\color{mycyan}},
    % errors (also cyan emph)
    emph={[6]Exception,NameError,IndexError,SyntaxError,TypeError,ValueError,OverflowError,ZeroDivisionError},
    emphstyle={[6]\color{mycyan}},
    % errors (also cyan emph)
    emph={[7]copy,deepcopy,append,real,imag},
    emphstyle={[7]\color{black}},
    % 
    showstringspaces=false,
	breaklines=true,
	numbers=left,
    frame=tb,
	xleftmargin=15pt
}}

\newcommand\javascriptstyle{\lstset{
	basicstyle=\small\sffamily,
    keywordstyle=\color{blue}\bfseries,
    commentstyle=\color{purple}\ttfamily,
    stringstyle=\color{red}\ttfamily,
	numberstyle=\color{gray},
    keywords={typeof, new, true, false, catch, function, return, null, catch, switch, var, if, in, while, do, else, case, break},
    ndkeywords={class, export, boolean, throw, implements, import, this},
    ndkeywordstyle=\color{darkgray}\bfseries,
    identifierstyle=\color{black},
    comment=[l]{//},
    morecomment=[s]{/*}{*/},
    morestring=[b]',
    morestring=[b]",
    showstringspaces=false,
	breaklines=true,
	numbers=left,
    frame=tb,
	xleftmargin=15pt,
    sensitive=false,
}}

\newcommand\csharpstyle{\lstset{
	language=csh,
	tabsize=4,
	basicstyle=\small\sffamily,
	numberstyle=\color{gray},
	stringstyle=\color{myreddark},
    commentstyle=\color{mygreen},
	morecomment=[l]{//}, %use comment-line-style!
	morecomment=[s]{/*}{*/}, %for multiline comments
    % KEYWORDS
	keywordstyle=\normalsize\color{myblue},%\bfseries,
	morekeywords={ abstract, event, new, struct,
		as, explicit, null, switch,
		base, extern, object, this,
		bool, false, operator, throw,
		break, finally, out, true,
		byte, fixed, override, try,
		case, float, params, typeof,
		catch, for, private, uint,
		char, foreach, protected, ulong,
		checked, goto, public, unchecked,
		class, if, readonly, unsafe,
		const, implicit, ref, ushort,
		continue, in, return, using,
		decimal, int, sbyte, virtual,
		default, interface, sealed, volatile,
		delegate, internal, short, void,
		do, is, sizeof, while,
		double, lock, stackalloc,
		else, long, static,
		enum, namespace, string},
	% 
    showstringspaces=false,
	breaklines=true,
	numbers=left,
    frame=tb,
	xleftmargin=15pt	
}}

% Python environment
\lstnewenvironment{python}[1][]
{
	\pythonstyle
	\lstset{#1}
}
{}
\lstnewenvironment{javascript}[1][]
{
	\javascriptstyle
	\lstset{#1}
}
{}
\lstnewenvironment{csharp}[1][]
{
	\csharpstyle
	\lstset{#1}
}
{}

% CODE FOR EXTERNAL FILES
\newcommand\pythonexternal[2][]{{
		\pythonstyle
		\lstinputlisting[#1]{#2}}}

% CODE FOR INLINE
\newcommand\pythoninline[1]{{\pythonstyle\lstinline!#1!}}
\newcommand\csharpinline[1]{{\csharpstyle\lstinline!#1!}}
\newcommand\javascriptinline[1]{{\javascriptstyle\lstinline!#1!}}

%% DEFINE CUSTOM COMMANDS AND SHORTCUTS

% some commands
\def\ba#1\ea{\begin{align}#1\end{align}}
\def\bas#1\eas{\begin{align*}#1\end{align*}}
\def\bmat#1\emat{\begin{pmatrix}#1\end{pmatrix}}
\newcommand{\ve}[1]{\vec{#1}}
\newcommand{\veTwo}[2]{\begin{pmatrix}#1\\#2\end{pmatrix}}
\newcommand{\veThree}[3]{\begin{pmatrix}#1\\#2\\#3\end{pmatrix}}
\newcommand{\veFour}[4]{\begin{pmatrix}#1\\#2\\#3\\#4\end{pmatrix}}
\newcommand{\ora}[1]{\overrightarrow{#1}}
\newcommand{\ola}[1]{\overleftarrow{#1}}
\newcommand{\s}[1]{\sqrt{#1}}

\def \bit{\begin{itemize}\setlength\itemsep{0em}} %\vspace{-5mm}
	\def \eit{\end{itemize}}

\def \ben{\begin{enumerate}\setlength\itemsep{0em}} %\vspace{-5mm}
	\def \een{\end{enumerate}}

\def \N{\mathbb{N}}
\def \Z{\mathbb{Z}}
\def \Q{\mathbb{Q}}
\def \R{\mathbb{R}}
\def \C{\mathbb{C}}

\def \ra{\rightarrow}
\def \longra{\longrightarrow}
\def \Ra{\Rightarrow}
\def \Longra{\Longrightarrow}
\def \la{\leftarrow}
\def \longla{\longleftarrow}
\def \La{\Leftarrow}
\def \Longla{\Longleftarrow}

\def \lra{\leftrightarrow}
\def \longlra{\longleftrightarrow}
\def \Lra{\Leftrightarrow}
\def \Longlra{\Longleftrightarrow}

\def \l{\left}
\def \r{\right}

\def \a{\alpha}
\def \b{\beta}
\def \g{\gamma}

\def \c{\cdot}

\def \el{\in}
\def \notel{\notin}

\newcommand{\f}{\frac}

\def \q{\quad}
\newcommand{\p}{\phantom}

\def\bs#1\es{\begin{split}#1\end{split}}


\def \L{\mathbb{L}}




\begin{document}

%%%%%%%%%%%%%%%%%%%%%%%%%%%%%%%%%%%%%%%%%%%%%%%%%%%%%%%
%%%%%%%%%%%%%%%%%%%%%%%%%%%%%%%%%%%%%%%%%%%%%%%%%%%%%%%
%%%%%%%%%%%%%%%%%%%%%%%%%%%%%%%%%%%%%%%%%%%%%%%%%%%%%%%

%%% AB HIER ARBEITEN, WEITER OBEN NICHTS VERAENDERN
%%% Ausnahme: Einbinden weitere Pakete

\selectlanguage{ngerman} %% HIER SPRACHE EINSTELLEN! english, ngerman

\begin{titlepage}
	\clearpage\thispagestyle{empty}	
	\setstretch{1}
	
	\begin{minipage}[t]{\textwidth}
		\begin{minipage}[t]{0.5\textwidth}
			Zeynep Zülal Keskin\\
			Mattfeldstrasse 6\\
			9532 Rickenbach b. Wil\\
			077 277 43 77\\
			zekeskin@ksr.ch
		\end{minipage}
		\begin{minipage}[t]{0.5\textwidth}
			\begin{flushright}
				Kantonsschule Romanshorn\\
				Klasse 4Ma\\
				Maturaarbeit
			\end{flushright}
		\end{minipage}
	\end{minipage}
	
	\vspace{4cm}
	{
		\centering
		\Huge\bfseries N-Körper Problem\par
		\vspace{1cm}
		\includegraphics[width=0.15\textwidth]{example-image-1x1}\par
	}
	
	\vspace{9cm}	
	\noindent
    Fach: Physik, Mathematik, Informatik \noindent\\
	Betreuungsperson: Dr. Andreas Schärer\\
	Abgabetermin: 6. Januar 2025
	
\end{titlepage}

%%%%%%%%%%%%%%%%%%%%%%%%%%%%%%%%%%%%%%%%%%%%%%%%%%%%%%%
%%%%%%%%%%%%%%%%%%%%%%%%%%%%%%%%%%%%%%%%%%%%%%%%%%%%%%%
%%%%%%%%%%%%%%%%%%%%%%%%%%%%%%%%%%%%%%%%%%%%%%%%%%%%%%%

% \null\thispagestyle{empty}\clearpage

%%% ABSTRACT

\pagenumbering{roman}

\section*{Abstract}

Das N-Körper-Problem ist ein fundamentales Problem der Himmelsmechanik, das die Bewegung von N Massenpunkten unter gegenseitiger gravitativer Anziehung beschreibt. Diese Arbeit untersucht zwei verschiedene Simulationsmethoden für das N-Körper-Problem: die direkte Methode und den Barnes-Hut-Algorithmus. Ziel ist es, die Genauigkeit und Effizienz dieser Methoden zu analysieren und zu vergleichen.
%%%%%%%%%%%%%%%%%%%%%%%%%%%%%%%%%%%%%%%%%%%%%%%%%%%%%%%
%%%%%%%%%%%%%%%%%%%%%%%%%%%%%%%%%%%%%%%%%%%%%%%%%%%%%%%
%%%%%%%%%%%%%%%%%%%%%%%%%%%%%%%%%%%%%%%%%%%%%%%%%%%%%%%

%%% TABLE OF CONTENTS

%\null\thispagestyle{empty}\clearpage
\newpage
%\setcounter{page}{1}
\tableofcontents

\parindent=0pt
\parskip=6pt

\newpage

\pagenumbering{arabic}

%%%%%%%%%%%%%%%%%%%%%%%%%%%%%%%%%%%%%%%%%%%%%%%%%%%%%%%
%%%%%%%%%%%%%%%%%%%%%%%%%%%%%%%%%%%%%%%%%%%%%%%%%%%%%%%
%%%%%%%%%%%%%%%%%%%%%%%%%%%%%%%%%%%%%%%%%%%%%%%%%%%%%%%

\section{Einleitung}
Diese Arbeit befasst sich mit einem der grundlegendsten und komplexesten Probleme der Himmelsmechanik und -dynamik: dem N-Körper-Problem in der Physik. Dieses Problem ist nicht nur von theoretischem Interesse, sondern findet auch praktische Anwendungen in der Astronomie, Raumfahrt und vielen anderen Bereichen der Physik. Die zentrale Forschungsfrage dieser Arbeit lautet: Welche mathematischen und physikalischen Methoden können verwendet werden, um die Bewegung von N-Körpern zu beschreiben und zu verstehen? 
Das Ziel dieser Arbeit ist es, die Bewegung von N-Körpern zu definieren und umfassend zu analysieren. Dabei soll der physikalische Hintergrund dieses Problems beleuchtet und seine historische Entwicklung nachgezeichnet werden. Ein besonderer Schwerpunkt liegt auf der Darstellung und Untersuchung der verschiedenen mathematischen und physikalischen Methoden, die im Laufe der Zeit entwickelt wurden, um die Dynamik von N-Körpern zu erklären.
Darüber hinaus werden zwei detaillierte Simulationen entwickelt, um die theoretischen Konzepte praktisch zu veranschaulichen. Diese Simulationen dienen nicht nur der Erklärung des Problems, sondern auch dem Vergleich verschiedener Lösungsansätze. Durch diese Herangehensweise soll ein tieferes Verständnis für die Komplexität des N-Körper-Problems und die Herausforderungen bei seiner Lösung vermittelt werden.
Ein weiterer Aspekt dieser Arbeit ist die Untersuchung der praktischen Relevanz des N-Körper-Problems. Dabei wird aufgezeigt, wie die theoretischen Erkenntnisse in der Astronomie zur Berechnung von Planetenbahnen, in der Raumfahrt zur Planung von Satellitenmissionen und in anderen Bereichen der Physik zur Modellierung komplexer Systeme angewendet werden können.
Zusammenfassend strebt diese Arbeit an, einen umfassenden Überblick über das N-Körper-Problem zu geben, seine Bedeutung in der modernen Physik aufzuzeigen und durch die Entwicklung und den Vergleich von Simulationen einen Beitrag zur Lösung dieses faszinierenden und herausfordernden Problems zu leisten.

\newpage
\section{Theoretische Grundlagen}

\subsection{Problembeschreibung und historische Entwicklung}
Das N-Körper-Problem beschäftigt sich mit der dynamischen Analyse von N Massepunkten, die durch ihre gegenseitige gravitative Anziehung beeinflusst werden. Während das Zwei-Körper-Problem durch Newtons Gesetz der universellen Gravitation und die Keplerschen Gesetze exakt gelöst werden kann, wird das N-Körper-Problem für N Körper wesentlich komplexer und zeigt in der Regel chaotisches Verhalten. Die historische Entwicklung dieses Problems begann im 17. Jahrhundert mit den grundlegenden Arbeiten von Johannes Kepler und Isaac Newton, die die Basis für die Himmelsmechanik schufen. Im 18. und 19. Jahrhundert erweiterten Joseph-Louis Lagrange und Pierre-Simon Laplace das Verständnis durch die Einführung der Lagrangeschen Gleichungen und durch die Analyse spezieller Lösungen wie der Lagrange-Punkte, die besondere Gleichgewichtslagen in einem gravitativen System beschreiben. Henri Poincaré entdeckte Ende des 19. Jahrhunderts die chaotische Natur des Systems, was den Beginn der modernen Chaosforschung einleitete. Im 20. und 21. Jahrhundert ermöglichten Fortschritte in der Computertechnologie und numerischen Methoden präzise Simulationen und Analysen komplexer N-Körper-Systeme, die wichtige Einblicke in die Dynamik von Galaxien und die Planung von Raumfahrtmissionen bieten.

\subsection{Gravitationskraft}
Die Gravitationskraft $\vec{F}_{12}$ zwischen zwei punktförmigen Massen $m_1$ und $m_2$ liegt auf der Verbindungslinie der beiden Massen. Der Betrag $F$ der Gravitationskraft ist proportional zu den Massen $m_1$ und $m_2$ sowie umgekehrt proportional zum Quadrat des Abstands $r$ (die relative Position ist die Position eines Körpers relativ zu einem anderen: \(\mathbf{r} = \mathbf{r}_2 - \mathbf{r}_1\)) der Massen. Er berechnet sich durch:
\[
F = G \frac{m_1 m_2}{r^2}
\]
mit der Gravitationskonstante $G = \SI{6.674e-11}{\frac{m^3}{kg\,s^2}}$.

\subsection{Die Keplerschen Gesetze}
Die Keplerschen Gesetze beschreiben die Bewegung von Planeten um die Sonne und sind grundlegende Prinzipien der Himmelsmechanik. Johannes Kepler formulierte sie im frühen 17. Jahrhundert auf Basis der Beobachtungen von Tycho Brahe. Diese Gesetze markierten einen Wendepunkt von der klassischen, kreisförmigen Vorstellung der Planetenbewegung zu einem elliptischen Modell. Im Folgenden werde ich die drei Keplerschen Gesetze erläutern.
\subsubsection{1. Keplersches Gesetz: Das Gesetz der Ellipsen}
Das erste Keplersche Gesetz besagt:
\begin{quote}
    Die Bahnen der Planeten um die Sonne sind Ellipsen, wobei sich die Sonne in einem der beiden Brennpunkte befindet.
\end{quote}
Mathematisch lässt sich eine Ellipse durch die Gleichung
\[
\frac{x^2}{a^2} + \frac{y^2}{b^2} = 1
\]
beschreiben, wobei \(a\) die große Halbachse und \(b\) die kleine Halbachse der Ellipse ist. Der Planet bewegt sich auf einer solchen Ellipsenbahn, wobei sich die Sonne nicht im Zentrum der Ellipse befindet, sondern in einem der beiden Brennpunkte. Dies war ein entscheidender Unterschied zur zuvor angenommenen Kreisbahn der Planeten.
\subsubsection{2. Keplersches Gesetz: Das Gesetz der Flächen}
Das zweite Keplersche Gesetz lautet:
\begin{quote}
    Der Fahrstrahl (die Verbindungslinie) zwischen einem Planeten und der Sonne überstreicht in gleichen Zeitspannen gleiche Flächen.
\end{quote}
Dies bedeutet, dass sich ein Planet auf seiner elliptischen Bahn unterschiedlich schnell bewegt: Er ist schneller, wenn er sich näher an der Sonne befindet (Perihel) und langsamer, wenn er weiter entfernt ist (Aphel). Dies lässt sich mathematisch durch die Erhaltung des Drehimpulses ausdrücken, da der Winkelgeschwindigkeitsvektor und der Abstand zur Sonne in Beziehung stehen. Formal lässt sich dies auch durch die Flächengeschwindigkeit beschreiben:
\[
\frac{dA}{dt} = \text{konstant},
\]
wobei \(A\) die Fläche und \(t\) die Zeit ist. Diese Konstanz der Flächengeschwindigkeit ist ein Hinweis auf das Prinzip der Erhaltung des Drehimpulses.
\subsubsection{3. Keplersches Gesetz: Das Harmoniegesetz}
Das dritte Keplersche Gesetz beschreibt die Beziehung zwischen der Umlaufzeit eines Planeten und seiner Entfernung von der Sonne:
\begin{quote}
    Die Quadrate der Umlaufzeiten zweier Planeten verhalten sich wie die Kuben der großen Halbachsen ihrer Bahnen.
\end{quote}
Mathematisch wird dies durch die Gleichung
\[
\frac{T_1^2}{T_2^2} = \frac{a_1^3}{a_2^3}
\]
ausgedrückt, wobei \(T_1\) und \(T_2\) die Umlaufzeiten der beiden Planeten und \(a_1\) und \(a_2\) die großen Halbachsen ihrer Bahnen sind. Dies zeigt, dass Planeten, die weiter von der Sonne entfernt sind, eine längere Umlaufzeit haben als solche, die sich näher an der Sonne befinden.

\subsection{Zwei-Körper-Problem}

Das Zwei-Körper-Problem beschreibt in der klassischen Mechanik die Bewegung zweier Körper, die sich unter dem Einfluss einer zentralen Kraft (in der Regel die Gravitation) gegenseitig anziehen. Es ist ein fundamentales Problem der Himmelsmechanik und bildet die Grundlage für das Verständnis von Bahnbewegungen, wie sie z.B. bei Planeten, Satelliten oder Doppelsternsystemen auftreten.
Gegeben sind zwei Massen \( m_1 \) und \( m_2 \), die sich unter dem Einfluss ihrer gegenseitigen Gravitationskraft bewegen.
Ziel ist es, die Bewegungsgleichungen für beide Körper zu lösen und ihre Bahnen zu bestimmen.
\subsubsection{Schwerpunktsystem und Reduktion auf das Ein-Körper-Problem}
Um die Lösung des Problems zu vereinfachen, führt man ein Koordinatensystem ein, das den Schwerpunkt des Systems beschreibt. Der Schwerpunkt befindet sich an der Position:
\[
R_{\text{Schwerpunkt}} = \frac{m_1 r_1 + m_2 r_2}{m_1 + m_2}
\]
Durch die Transformation in das Schwerpunktsystem kann die Bewegung der beiden Körper auf die Bewegung eines Körpers mit der sogenannten reduzierten Masse \( \mu \) relativ zu einem der beiden Körper reduziert werden. Die reduzierte Masse wird durch:
\[
\mu = \frac{m_1 m_2}{m_1 + m_2}
\]
definiert. Die Bewegung dieses fiktiven Körpers wird durch eine Differenzialgleichung beschrieben, die sich aus dem zweiten Newtonschen Gesetz ableiten lässt:
\[
\mu \frac{d^2}{dt^2} = - G \frac{m_1 m_2}{r^2} 
\]
Dies ist die Grundgleichung des Zwei-Körper-Problems, die die Bewegung eines Körpers unter dem Einfluss der Gravitationskraft eines anderen Körpers beschreibt.


\subsubsection{Lösung der Bahnkurve}
Im Zweikörperproblem beschreibt der Abstand \( r \) die Distanz zwischen den beiden Massen \( m_1 \) und \( m_2 \) oder die relative Position eines Körpers zum Schwerpunkt des Systems. Der Winkel \( \theta \) gibt die Lage des Körpers auf seiner Umlaufbahn in Bezug auf eine feste Referenzachse an. In Polarkoordinaten wird die Bahn des Körpers durch die Gleichung:
\[
r(\theta) = \frac{p}{1 + e \cos(\theta)}
\]
beschrieben, wobei \( p \) der Bahnhalbmesser und \( e \) die Exzentrizität (ist ein Maß für die Abweichung von einer perfekten Kreisbahn: Sie liegt zwischen 0 (Kreis) und größer als 1 (Hyperbel). Diese Information ist entscheidend, um die Form und die Eigenschaften der Bahn zu verstehen) ist. Diese Parameter charakterisieren, ob die Bahn elliptisch, parabolisch oder hyperbolisch ist, wobei der Winkel \( \theta \) die Bewegung entlang der Bahn angibt.

\begin{figure}[H]
	\centering
	\includegraphics[width=0.4\textwidth]{Polarkoordinaten.svg.png}
	\caption[Eintrag in Abbildungsverzeichnis von Polarkoordinatsystem]{Polar koordinaten sind ein System zur Beschreibung der Position eines Punktes durch seinen Abstand vom Ursprung und den Winkel, den die Verbindungsgerade zum Ursprung mit einer festen Referenzachse bildet.}
	\label{polarkoordinatsystem.}
\end{figure}

\subsection{Drei-Körper-Problem}
Im Gegensatz zum Zwei-Körper-Problem, das durch die Kepler-Gesetze vollständig analytisch lösbar ist, gibt es für das Drei-Körper-Problem keine allgemeine analytische Lösung. Das Problem beschreibt die Bewegung dreier Objekte (z.\,B.\ Sterne, Planeten oder Monde), die sich gegenseitig aufgrund der Gravitation anziehen. Jedes Objekt wird durch die resultierende Gravitationskraft der beiden anderen beeinflusst, was zu einer sehr komplizierten Dynamik führt.
Für das Zwei-Körper-Problem kann die Bewegung der beiden Objekte auf elliptische, parabolische oder hyperbolische Bahnen reduziert werden, je nach den Anfangsbedingungen. Das Drei-Körper-Problem hingegen führt oft zu unvorhersehbaren, chaotischen Bewegungen, außer in speziellen Fällen mit symmetrischen oder stark eingeschränkten Anfangsbedingungen.

\subsubsection{Mathematische Beschreibung}
Das Drei-Körper-Problem befasst sich mit der Bestimmung der Bewegung von drei Massen \( m_1 \), \( m_2 \) und \( m_3 \), die durch ihre wechselseitigen Gravitationskräfte miteinander interagieren. Die Beschleunigung jeder dieser Körper wird durch die Gravitationsanziehung der beiden anderen Massen bestimmt, gemäß dem Gravitationsgesetz von Newton. Die Position jedes Körpers wird durch seinen Vektor \( \mathbf{r}_i \) im dreidimensionalen Raum beschrieben. Die Gravitationskraft zwischen zwei Körpern nimmt mit der dritten Potenz des Abstands zwischen ihnen ab, wobei die resultierende Kraft auf jeden Körper die Summe der von den beiden anderen Körpern ausgeübten Kräfte ist. Dies führt zu einem System gekoppelter, nichtlinearer Differentialgleichungen, bei dem die Beschleunigung jedes Körpers von den relativen Positionen aller drei Massen abhängt. Aufgrund der Komplexität dieses Systems ist eine exakte analytische Lösung im Allgemeinen nicht möglich, weshalb es in der Regel numerisch gelöst wird.
Die wichtigsten Gleichungen, die die Bewegung der drei Körper beschreiben, sind:
\[
\ddot{\mathbf{r}}_1 = -G m_2 \frac{\mathbf{r}_1 - \mathbf{r}_2}{|\mathbf{r}_1 - \mathbf{r}_2|^3} - G m_3 \frac{\mathbf{r}_1 - \mathbf{r}_3}{|\mathbf{r}_1 - \mathbf{r}_3|^3}
\]
\[
\ddot{\mathbf{r}}_2 = -G m_3 \frac{\mathbf{r}_2 - \mathbf{r}_3}{|\mathbf{r}_2 - \mathbf{r}_3|^3} - G m_1 \frac{\mathbf{r}_2 - \mathbf{r}_1}{|\mathbf{r}_2 - \mathbf{r}_1|^3}
\]
\[
\ddot{\mathbf{r}}_3 = -G m_1 \frac{\mathbf{r}_3 - \mathbf{r}_1}{|\mathbf{r}_3 - \mathbf{r}_1|^3} - G m_2 \frac{\mathbf{r}_3 - \mathbf{r}_2}{|\mathbf{r}_3 - \mathbf{r}_2|^3}
\]

Diese Gleichungen beschreiben, wie sich die Beschleunigungen der drei Körper aufgrund ihrer gegenseitigen Gravitationskräfte entwickeln.

\subsubsection{Sonderfälle}

\subsubsection{Kollinear}
Im 1767 fand Leonhard Euler drei Familien von periodischen Lösungen, bei denen die drei Massen zu jedem Zeitpunkt kollinear sind. In der Eulerschen Lösung des Dreikörperproblems beginnen die Körper auf einer Linie. Sie können unterschiedliche Massen haben und ihre Bewegung wird nur durch ihre gegenseitigen anziehenden Gravitationskräfte angetrieben, die alle auf den Massenschwerpunkt des Systems gerichtet sind. Die Bedingungen für die Verwirklichung der Eulerschen Lösung sind wie folgt: 
\subsubsection{Dreieck}
\subsubsection{Figure-8}
\subsubsection{Runge Kutta Methode}


\subsection{N-Körper Problem}
Das n-Körper-Problem ist ein klassisches Problem in der Physik und beschreibt die Bewegung mehrerer Massenpunkte, die sich gegenseitig durch ihre Gravitationskräfte beeinflussen. Die Grundgleichung für die Beschleunigung eines Körpers \( j \) in einem solchen System ist gegeben durch:

\[
m_j \ddot{\mathbf{q}}_j = G \sum_{\substack{k=1 \\ k \neq j}}^n \frac{m_j m_k (\mathbf{q}_k - \mathbf{q}_j)}{|\mathbf{q}_k - \mathbf{q}_j|^3}
\]

Hier steht \( m_j \) für die Masse des betrachteten Körpers, \( \mathbf{q}_j \) für dessen Position, und \( G \) für die Gravitationskonstante. Die Summe erstreckt sich über alle anderen Körper \( k \) (mit \( k \neq j \)), die eine Gravitationskraft auf den Körper \( j \) ausüben. Der Ausdruck \( \frac{\mathbf{q}_k - \mathbf{q}_j}{|\mathbf{q}_k - \mathbf{q}_j|^3} \) beschreibt dabei die Richtung und Stärke der Gravitationskraft zwischen den beiden Körpern \( j \) und \( k \), wobei der Abstand \( |\mathbf{q}_k - \mathbf{q}_j| \) hoch drei im Nenner steht, um sicherzustellen, dass die Kraft gemäß dem Gravitationsgesetz mit dem Quadrat des Abstands abnimmt.

Das n-Körper-Problem stellt eine Herausforderung dar, weil ab mehr als zwei Körpern eine chaotische Dynamik entsteht, die eine exakte analytische Lösung unmöglich macht. Während das Zwei-Körper-Problem stabile, vorhersagbare Bahnen liefert, wird das System bei drei oder mehr Körpern extrem empfindlich gegenüber Anfangsbedingungen.
Dadurch sind die Bewegungen der Körper langfristig unvorhersehbar und schwer exakt zu berechnen. Numerische Simulationen, die die Bewegung der Körper Schritt für Schritt berechnen, bieten eine Lösungsmöglichkeit, jedoch nur in Form von Näherungen. Die Berechnungen sind dabei extrem aufwendig, da für jede Wechselwirkung die Kräfte separat berücksichtigt werden müssen, was bei steigender Körperanzahl exponentiell mehr Rechenleistung erfordert.


\section{Material und Methoden}

\subsection{Der Barnes-Hut-Algorithmus}
Der Barnes-Hut-Algorithmus ist ein effizientes Näherungsverfahren zur Berechnung der Kräfte in N-Körper-Problemen, insbesondere in der Astrophysik. Er wurde 1986 von Josh Barnes und Piet Hut entwickelt und reduziert den Rechenaufwand von $O(N^2)$ auf $O(N \log N)$, wodurch Simulationen mit einer großen Anzahl von Teilchen ermöglicht werden.
Der Algorithmus basiert auf der Annahme, dass Gruppen von weit entfernten Teilchen als einzelne Massepunkte behandelt werden können. Dazu wird der Raum rekursiv in immer kleinere Bereiche unterteilt, bis jeder Bereich entweder leer ist oder genau ein Teilchen enthält. Diese Struktur wird in Form eines Baumes, genauer eines Quadtrees im zweidimensionalen Raum oder eines Octrees im dreidimensionalen Raum, dargestellt.

\subsubsection{Algorithmusschritte}

\paragraph{1. Baumerstellung}
Der gesamte Raum wird in vier (2D) bzw. acht (3D) Quadranten unterteilt. Jedes Teilchen wird entsprechend seiner Position in einen dieser Quadranten einsortiert. Ist ein Quadrant bereits belegt, wird er weiter unterteilt, bis jedes Teilchen in einem eigenen Unterquadranten liegt. Dieses rekursive Unterteilen führt zur Bildung des Quadtrees bzw. Octrees.

\paragraph{2. Masseverteilung}
Nach der Baumerstellung wird für jeden Knoten des Baumes die Gesamtmasse und der Massenschwerpunkt der in ihm enthaltenen Teilchen berechnet. Diese Informationen sind entscheidend für die spätere Kraftberechnung.

\paragraph{3. Kraftberechnung}
Um die auf ein bestimmtes Teilchen wirkende Kraft zu berechnen, wird der Baum erneut durchlaufen. Für jeden Knoten wird das Verhältnis von dessen Größe zur Entfernung zum betrachteten Teilchen berechnet. Ist dieses Verhältnis kleiner als ein vorgegebener Schwellenwert ($\theta$), wird der gesamte Knoten als eine einzelne Masse im Massenschwerpunkt betrachtet. Andernfalls werden die Unterknoten weiter untersucht. Dieses Verfahren reduziert die Anzahl der notwendigen Kraftberechnungen erheblich.

\subsubsection{Parameter $\theta$}
Der Schwellenwert $\theta$ bestimmt die Genauigkeit der Näherung. Ein kleiner Wert für $\theta$ führt zu einer höheren Genauigkeit, da weniger Gruppierungen vorgenommen werden und mehr direkte Berechnungen stattfinden. Ein größerer Wert reduziert den Rechenaufwand, kann jedoch die Genauigkeit beeinträchtigen. Typischerweise wird $\theta$ auf Werte um 0,5 bis 1 gesetzt, abhängig von den Anforderungen an die Simulation.

\subsubsection{Vorteile und Anwendungen}
Der Barnes-Hut-Algorithmus ermöglicht die Simulation von Systemen mit Millionen von Teilchen, was mit direkten Berechnungsmethoden unpraktikabel wäre. Er findet Anwendung in der Astrophysik, beispielsweise bei der Simulation von Galaxien, sowie in anderen Bereichen wie der Molekulardynamik und der Informatik, etwa beim kräftebasierten Zeichnen von Graphen.

\subsection{React.js -- Funktionsweise und Anwendungen}

React.js ist eine JavaScript-Bibliothek, die zur Entwicklung von Benutzeroberflächen verwendet wird. Sie basiert auf einer komponentenbasierten Architektur, bei der Anwendungen in wiederverwendbare, eigenständige Komponenten unterteilt werden. Jede Komponente verwaltet ihre eigenen Daten und ihr Verhalten. Dies erleichtert die Pflege und Erweiterung komplexer Anwendungen. Ein zentrales Merkmal von React.js ist das virtuelle DOM. Statt Änderungen direkt im echten DOM vorzunehmen, werden diese zunächst im virtuellen DOM simuliert. Anschließend werden nur die tatsächlich geänderten Elemente in den echten DOM übernommen. Dieser Ansatz verbessert die Leistung erheblich, insbesondere bei Anwendungen mit häufigen Aktualisierungen.

Ein weiteres wichtiges Konzept in React.js ist der unidirektionale Datenfluss. Daten werden von Eltern- zu Kinderkomponenten weitergegeben, was die Vorhersagbarkeit und Wartbarkeit des Codes erhöht. Seit der Einführung von React Hooks in Version 16.8 können Entwickler Zustände und Lebenszyklusmethoden in funktionalen Komponenten verwenden. Hooks wie \texttt{useState} und \texttt{useEffect} vereinfachen die Verwaltung von Zuständen und Nebenwirkungen.

\subsubsection{Anwendungen von React.js}
React.js eignet sich besonders gut für Single-Page Applications (SPAs). In solchen Anwendungen werden nur die benötigten Inhalte aktualisiert, anstatt komplette Seiten neu zu laden. Dies führt zu einer flüssigen und reaktionsschnellen Benutzererfahrung. Darüber hinaus ist React.js ideal für datenintensive Anwendungen, wie z.B. Dashboards und Simulationen, die häufige Aktualisierungen und eine effiziente Verwaltung des Zustands erfordern. Weitere Anwendungen finden sich in der Datenvisualisierung, insbesondere in Kombination mit Bibliotheken wie D3.js und Chart.js, sowie in interaktiven Simulationen und Modellen.

\subsubsection{Warum eignet sich React.js für den Barnes-Hut-Algorithmus?}
React.js bietet eine ideale Grundlage für die Implementierung des Barnes-Hut-Algorithmus, da dieser auf häufigen Aktualisierungen und dynamischen Daten basiert. Das virtuelle DOM sorgt dafür, dass nur die tatsächlich betroffenen Teile der Benutzeroberfläche aktualisiert werden. Dies ist besonders wichtig bei der Visualisierung von Simulationen mit vielen Teilchen, bei denen laufend neue Positionen und Kräfte berechnet werden.

Die komponentenbasierte Struktur von React.js ermöglicht es, jede Simulationseinheit -- sei es ein Teilchen oder ein Knoten im Baum -- als eigenständige Komponente zu modellieren. Dies erleichtert die Verwaltung von Zuständen und Interaktionen innerhalb der Simulation. Mit React Hooks wie \texttt{useState} und \texttt{useEffect} können diese Zustände effizient aktualisiert werden, ohne den gesamten Baum neu zu rendern.

Ein weiterer Vorteil ist die einfache Integration von Bibliotheken zur Datenvisualisierung und physikalischen Modellierung. Bibliotheken wie D3.js oder Three.js können nahtlos in React-Projekte eingebunden werden, um komplexe Darstellungen und Animationen zu erzeugen. Darüber hinaus unterstützt React.js den Einsatz von WebGL, was eine hardwarebeschleunigte Darstellung von Simulationen ermöglicht.

Zusammenfassend bietet React.js durch seine Effizienz, Modularität und Erweiterbarkeit eine ideale Plattform zur Implementierung von Simulationen auf Basis des Barnes-Hut-Algorithmus. Dies macht es zu einer bevorzugten Wahl für Anwendungen, die sowohl hohe Leistung als auch eine interaktive Benutzeroberfläche erfordern.

\subsection{Vorgehensweise und Implementierung}

\section{Resultat}
\subsection{Vergleich der Simulationsergebnisse}

\section{Diskussion}
\subsection{Vorteile der verwendeten Algorithmen}
\subsection{Herausforderungen und Limitierungen}
\subsection{Vergleich mit bestehenden Methoden}

\section{Schlussfolgerung}
\section{Literaturverzeichnis}
\section{Abbildungsverzeichnis}

\end{document}

